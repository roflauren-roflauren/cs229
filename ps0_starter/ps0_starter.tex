\documentclass{article}
\usepackage{amsmath,amssymb}
\usepackage{latexsym,amsthm,epsfig}
\usepackage{verbatim}
\usepackage{enumerate}
\usepackage{hyperref}
%\usepackage[english, activeacute]{babel}

% EM - MAP     11
% EM - NIPS    22
% PCA          14
% ICA          12
% RL - proof   16
% RL - imp     25


\usepackage{verbatim}
\usepackage{color}
\usepackage{bbm}
\usepackage{listings}
\usepackage{setspace}
\usepackage{float}
\definecolor{Code}{rgb}{0,0,0}
\definecolor{Decorators}{rgb}{0.5,0.5,0.5}
\definecolor{Numbers}{rgb}{0.5,0,0}
\definecolor{MatchingBrackets}{rgb}{0.25,0.5,0.5}
\definecolor{Keywords}{rgb}{0,0,1}
\definecolor{self}{rgb}{0,0,0}
\definecolor{Strings}{rgb}{0,0.63,0}
\definecolor{Comments}{rgb}{0,0.63,1}
\definecolor{Backquotes}{rgb}{0,0,0}
\definecolor{Classname}{rgb}{0,0,0}
\definecolor{FunctionName}{rgb}{0,0,0}
\definecolor{Operators}{rgb}{0,0,0}
\definecolor{Background}{rgb}{0.98,0.98,0.98}
\newcommand{\newsec}{\section}
\newcommand{\denselist}{\itemsep 0pt\partopsep 0pt}
\newcommand{\bitem}{\begin{itemize}\denselist}
\newcommand{\eitem}{\end{itemize}}
\newcommand{\benum}{\begin{enumerate}\denselist}
\newcommand{\eenum}{\end{enumerate}}

\newif\ifsolutions
\newcommand{\solutions}[2]{{
\ifsolutions
#2
\else
#1
\fi
}}

\newenvironment{labelledanswer}{{\bf Answer:} \sf \begingroup\color{red} }{\endgroup}%
\newcommand{\answer}[2]
{{
\ifsolutions
\begin{labelledanswer}
#2
\end{labelledanswer}
\else
#1
\fi
}}


\newcommand{\points}[1]{{\textbf{[#1 points]}}}
\newcommand{\subquestionpoints}[1]{{[#1 points]}}
\newcommand{\onePoint}{{\textbf{[1 point]}}}

\newcommand{\fig}[1]{\private{\begin{center}
{\Large\bf ({#1})}
\end{center}}}

\newcommand{\cpsf}[1]{{\centerline{\psfig{#1}}}}
\newcommand{\mytitle}[1]{\centerline{\LARGE\bf #1}}

\newcommand{\myw}{{\bf w}}

\newcommand{\mypar}[1]{\vspace{1ex}\noindent{\bf {#1}}}

\def\thmcolon{\hspace{-.85em} {\bf :} }

\newtheorem{THEOREM}{Theorem}[section]
\newenvironment{theorem}{\begin{THEOREM} \thmcolon }%
                        {\end{THEOREM}}
\newtheorem{LEMMA}[THEOREM]{Lemma}
\newenvironment{lemma}{\begin{LEMMA} \thmcolon }%
                      {\end{LEMMA}}
\newtheorem{COROLLARY}[THEOREM]{Corollary}
\newenvironment{corollary}{\begin{COROLLARY} \thmcolon }%
                          {\end{COROLLARY}}
\newtheorem{PROPOSITION}[THEOREM]{Proposition}
\newenvironment{proposition}{\begin{PROPOSITION} \thmcolon }%
                            {\end{PROPOSITION}}
\newtheorem{DEFINITION}[THEOREM]{Definition}
\newenvironment{definition}{\begin{DEFINITION} \thmcolon \rm}%
                            {\end{DEFINITION}}
\newtheorem{CLAIM}[THEOREM]{Claim}
\newenvironment{claim}{\begin{CLAIM} \thmcolon \rm}%
                            {\end{CLAIM}}
\newtheorem{EXAMPLE}[THEOREM]{Example}
\newenvironment{example}{\begin{EXAMPLE} \thmcolon \rm}%
                            {\end{EXAMPLE}}
\newtheorem{REMARK}[THEOREM]{Remark}
\newenvironment{remark}{\begin{REMARK} \thmcolon \rm}%
                            {\end{REMARK}}
%\newenvironment{proof}{\noindent {\bf Proof:} \hspace{.677em}}%
%                      {}

%theorem
\newcommand{\thm}{\begin{theorem}}
%lemma
\newcommand{\lem}{\begin{lemma}}
%proposition
\newcommand{\pro}{\begin{proposition}}
%definition
\newcommand{\dfn}{\begin{definition}}
%remark
\newcommand{\rem}{\begin{remark}}
%example
\newcommand{\xam}{\begin{example}}
%corollary
\newcommand{\cor}{\begin{corollary}}
%proof
\newcommand{\prf}{\noindent{\bf Proof:} }
%end theorem
\newcommand{\ethm}{\end{theorem}}
%end lemma
\newcommand{\elem}{\end{lemma}}
%end proposition
\newcommand{\epro}{\end{proposition}}
%end definition
\newcommand{\edfn}{\bbox\end{definition}}
%end remark
\newcommand{\erem}{\bbox\end{remark}}
%end example
\newcommand{\exam}{\bbox\end{example}}
%end corollary
\newcommand{\ecor}{\end{corollary}}
%end proof
\newcommand{\eprf}{\bbox\vspace{0.1in}}
%begin equation
\newcommand{\beqn}{\begin{equation}}
%end equation
\newcommand{\eeqn}{\end{equation}}

%\newcommand{\eqref}[1]{Eq.~\ref{#1}}

\newcommand{\KB}{\mbox{\it KB\/}}
\newcommand{\infers}{\vdash}
\newcommand{\sat}{\models}
\newcommand{\bbox}{\vrule height7pt width4pt depth1pt}

\newcommand{\act}[1]{\stackrel{{#1}}{\rightarrow}}
\newcommand{\at}[1]{^{(#1)}}

\newcommand{\argmax}{{\rm argmax}}
\newcommand{\V}{{\cal V}}
\newcommand{\C}{{\cal C}}
\newcommand{\calH}{{\cal H}}
\newcommand{\calL}{{\cal L}}
\newcommand{\calD}{{\cal D}}

\newcommand{\rimp}{\Rightarrow}
\newcommand{\dimp}{\Leftrightarrow}

\newcommand{\nf}{\bar{f}}
\newcommand{\ns}{\bar{s}}
\newcommand{\na}{\bar{a}}
\newcommand{\nh}{\bar{h}}
\newcommand{\nr}{\bar{r}}


\newcommand{\bX}{\mbox{\boldmath $X$}}
\newcommand{\bY}{\mbox{\boldmath $Y$}}
\newcommand{\bZ}{\mbox{\boldmath $Z$}}
\newcommand{\bU}{\mbox{\boldmath $U$}}
\newcommand{\bE}{\mbox{\boldmath $E$}}
\newcommand{\bx}{\mbox{\boldmath $x$}}
\newcommand{\be}{\mbox{\boldmath $e$}}
\newcommand{\by}{\mbox{\boldmath $y$}}
\newcommand{\bz}{\mbox{\boldmath $z$}}
\newcommand{\bu}{\mbox{\boldmath $u$}}
\newcommand{\bd}{\mbox{\boldmath $d$}}
\newcommand{\smbx}{\mbox{\boldmath $\scriptstyle x$}}
\newcommand{\smbd}{\mbox{\boldmath $\scriptstyle d$}}
\newcommand{\smby}{\mbox{\boldmath $\scriptstyle y$}}
\newcommand{\smbe}{\mbox{\boldmath $\scriptstyle e$}}

\newcommand{\norm}[1]{\left\|{#1}\right\|}
\newcommand{\ltwo}[1]{\norm{#1}_2}
\newcommand{\lone}[1]{\norm{#1}_1}
\newcommand{\linf}[1]{\norm{#1}_\infty}
\newcommand{\lfro}[1]{\norm{#1}_{\rm Fr}}
\newcommand{\matrixnorm}[1]{\left|\!\left|\!\left|{#1}\right|\!\right|\!\right|}

\newcommand{\hinge}[1]{\left[{#1}\right]_+}
\newcommand{\sign}{\mathop{\rm sign}}
\newcommand{\diag}{\mathop{\rm diag}}
\newcommand{\defeq}{:=}

\newcommand{\Parents}{\mbox{\it Parents\/}}
\newcommand{\B}{{\cal B}}

\newcommand{\word}[1]{\mbox{\it #1\/}}
\newcommand{\Action}{\word{Action}}
\newcommand{\Proposition}{\word{Proposition}}
\newcommand{\true}{\word{true}}
\newcommand{\false}{\word{false}}
\newcommand{\Pre}{\word{Pre}}
\newcommand{\Add}{\word{Add}}
\newcommand{\Del}{\word{Del}}
\newcommand{\Result}{\word{Result}}
\newcommand{\Regress}{\word{Regress}}
\newcommand{\Maintain}{\word{Maintain}}

\newcommand{\bor}{\bigvee}
\newcommand{\invert}[1]{{#1}^{-1}}

\newcommand{\commentout}[1]{}

\newcommand{\bmu}{\mbox{\boldmath $\mu$}}
\newcommand{\btheta}{\mbox{\boldmath $\theta$}}
\newcommand{\IR}{\mbox{$I\!\!R$}}

\newcommand{\tval}[1]{{#1}^{1}}
\newcommand{\fval}[1]{{#1}^{0}}

\newcommand{\tr}{{\rm tr}}
\newcommand{\vecy}{{\vec{y}}}
\renewcommand{\Re}{{\mathbb R}}
\newcommand{\R}{\mathbb{R}}
\newcommand{\half}{\frac{1}{2}}
\newcommand{\indic}[1]{\mathbf{1}\left\{{#1}\right\}}

\def\twofigbox#1#2{%
\noindent\begin{minipage}{\textwidth}%
\epsfxsize=0.35\maxfigwidth
\noindent \epsffile{#1}\hfill
\epsfxsize=0.35\maxfigwidth
\epsffile{#2}\\
\makebox[0.35\textwidth]{(a)}\hfill\makebox[0.35\textwidth]{(b)}%
\end{minipage}}

\def\twofigboxcd#1#2{%
\noindent\begin{minipage}{\textwidth}%
\epsfxsize=0.35\maxfigwidth
\noindent \epsffile{#1}\hfill
\epsfxsize=0.35\maxfigwidth
\epsffile{#2}\\
\makebox[0.35\textwidth]{(c)}\hfill\makebox[0.35\textwidth]{(d)}%
\end{minipage}}

\def\twofigboxnolabel#1#2{%
\begin{minipage}{\textwidth}%
\epsfxsize=0.35\maxfigwidth
\noindent \epsffile{#1}\hfill
\epsfxsize=0.35\maxfigwidth
\epsffile{#2}\\
%\makebox[0.48\textwidth]{(a)}\hfill\makebox[0.48\textwidth]{(b)}%
\end{minipage}
}

\def\threefigbox#1#2#3{%
\noindent\begin{minipage}{\textwidth}%
\epsfxsize=0.33\maxfigwidth
\noindent \epsffile{#1}\hfill
\epsfxsize=0.33\maxfigwidth
\noindent \epsffile{#2}\hfill 
\epsfxsize=0.33\maxfigwidth
\epsffile{#3}\\
\makebox[0.31\textwidth]{{\scriptsize (a)}}\hfill%
\makebox[0.31\textwidth]{{\scriptsize (b)}}\hfill
\makebox[0.31\textwidth]{{\scriptsize (c)}}%
\smallskip
\end{minipage}}

\def\threefigboxnolabel#1#2#3{%
\noindent\begin{minipage}{\textwidth}%
\epsfxsize=0.33\maxfigwidth
\noindent \epsffile{#1}\hfill
\epsfxsize=0.33\maxfigwidth
\noindent \epsffile{#2}\hfill 
\epsfxsize=0.33\maxfigwidth
\epsffile{#3}\\
%\makebox[0.31\textwidth]{{\scriptsize (a)}}\hfill%
%\makebox[0.31\textwidth]{{\scriptsize (b)}}\hfill
%\makebox[0.31\textwidth]{{\scriptsize (c)}}%
%\smallskip
\end{minipage}}

\newlength{\maxfigwidth}
\setlength{\maxfigwidth}{\textwidth}
%\def\captionsize {\footnotesize}
\def\captionsize {}

\newcommand{\xsi}{{x^{(i)}}}
\newcommand{\ysi}{{y^{(i)}}}
\newcommand{\xsj}{{x^{(j)}}}
\newcommand{\ysj}{{y^{(j)}}}
\newcommand{\zsi}{{z^{(i)}}}
\newcommand{\wsi}{{w^{(i)}}}
\newcommand{\esi}{{\epsilon^{(i)}}}
\newcommand{\calN}{{\cal N}}
\newcommand{\calX}{{\cal X}}
\newcommand{\calY}{{\cal Y}}
\newcommand{\ytil}{{\tilde{y}}}

\newcommand{\beas}{\begin{eqnarray*}}
\newcommand{\eeas}{\end{eqnarray*}}
\newcommand{\hhat}{{\hat{h}}}
\newcommand{\phat}{{\hat{p}}}
\newcommand{\ehat}{{\hat{\varepsilon}}}
\newcommand{\hstar}{{h^\ast}}

\newcommand{\Ber}{{\rm Bernoulli}}
\newcommand{\E}{{\rm E}}
\newcommand{\VC}{{\rm VC}}
\newcommand{\KL}{{\mathit KL}}
\newcommand{\Bernoulli}{{\rm Bernoulli}}

\newcommand{\Vstar}{{V^\ast}}
\newcommand{\Rmax}{{R_{\rm max}}}
\newcommand{\Bhat}{{\hat{B}}}


\newcommand{\ypr}{y^{(pr)}}
\newcommand{\zpr}{z^{(pr)}}
\newcommand{\xpr}{x^{(pr)}}
\newcommand{\epspr}{\epsilon^{(pr)}}
\newcommand{\yprsqr}{(\ypr)^2}
\newcommand{\zprsqr}{(\zpr)^2}
\newcommand{\xprsqr}{(\xpr)^2}
\newcommand{\epsprsqr}{(\epspr)^2}
\DeclareMathOperator{\Var}{Var}
\DeclareMathOperator{\Cov}{Cov}

\pagestyle{empty} \addtolength{\textwidth}{1.0in}
\addtolength{\textheight}{0.5in}
\addtolength{\oddsidemargin}{-0.5in}
\addtolength{\evensidemargin}{-0.5in}
\newcommand{\ruleskip}{\bigskip\hrule\bigskip}
\newcommand{\nodify}[1]{{\sc #1}}

\setlength{\parindent}{0pt} \setlength{\parskip}{0.5ex}
\setlength{\unitlength}{1cm}

\renewcommand{\Re}{{\mathbb R}}

\begin{document}

\pagestyle{myheadings} \markboth{}{CS229 Problem Set 0}


{\huge
\noindent CS 229, Spring 2023\\
Problem Set 0: Linear Algebra, Multivariable Calculus, and Probability\\
%{\Large Handout\solutions{\#12}{\#15} }
}

\solutionstrue

\ruleskip

{\bf Ungraded: optionally due Tuesday, April 12 at 11:59 pm on Gradescope.}

{\bf Notes:} \\ (1) These questions require thought, but do not require long
answers. Please be as concise as possible. \\ (2) If you have a question about
this homework, we encourage you to post your question on our Ed at \url{https://edstem.org/us/courses/37893/discussion/}.  \\ (3) If you missed the
first lecture or are unfamiliar with the collaboration or honor code policy,
please read the policy before you start. \\ (4) This specific homework is {\bf{\emph{not graded}}}, but we encourage you
to solve each of the problems to brush up on your linear algebra and probability.
Some of them may even be useful for subsequent problem sets. It also
serves as your introduction to using Gradescope for submissions. We strongly suggest you use LaTeX to write your problem set solutions (not only is it helpful for this class, but it is a good skill to learn). However, if you are scanning your document by cellphone, please use a scanning app such as
CamScanner. There will not be any late days allowed for this particular assignment.
\smallskip




{\bf Honor code:} 
We strongly encourage students to form study groups. Students may discuss and work on homework problems in groups. However, each student must write down the solution independently, and without referring to written notes from the joint session. Each student must understand the solution well enough in order to reconstruct it by him/herself. It is an honor code violation to copy, refer to, or look at written or code solutions from a previous year, including but not limited to: official solutions from a previous year, solutions posted online, and solutions you or someone else may have written up in a previous year. Furthermore, it is an honor code violation to post your assignment solutions online, such as on a public git repo. We run plagiarism-detection software on your code against past solutions as well as student submissions from previous years. Please take the time to familiarize yourself with the Stanford Honor Code 
\footnote{\url{https://communitystandards.stanford.edu/policies-and-guidance/honor-code}} and the Stanford Honor Code\footnote{\url{web.stanford.edu/class/archive/cs/cs106b/cs106b.1164/handouts/honor-code.pdf}} as it pertains to CS courses.

%We strongly encourage students to form study
%groups. Students may discuss and work on homework problems in
%groups. However, each student must write down the solutions independently,
%and without referring to written notes from the joint session.
%%In addition, each student should write on
%the problem set the set of people with whom s/he collaborated.
%Further, because we occasionally reuse problem set questions from previous
%years, we expect students not to copy, refer to, or look at the solutions
%in preparing their answers. It is an honor code violation to intentionally
%refer to a previous year's solutions.

\begin{enumerate}[1.]
\newpage
% -*- Mode: latex -*- %

\item \points{0} \textbf{Gradients and Hessians}

  Recall that a matrix $A \in \R^{n \times n}$ is \emph{symmetric} if
  $A^T = A$, that is, $A_{ij} = A_{ji}$ for all $i, j$. Also recall the
  gradient $\nabla f(x)$ of a function $f : \R^n \to \R$,
  which is the $n$-vector of partial derivatives
  \begin{equation*}
    \nabla f(x) = \left[\begin{matrix}
        \frac{\partial}{\partial x_1} f(x) \\ \vdots \\
        \frac{\partial}{\partial x_n} f(x) \end{matrix} \right]
    ~~ \mbox{where} ~~
    x = \left[\begin{matrix} x_1 \\ \vdots \\ x_n \end{matrix} \right].
  \end{equation*}
  The hessian $\nabla^2 f(x)$ of a function $f : \R^n \to \R$ is the
  $n \times n$ symmetric matrix of twice partial derivatives,
  \begin{equation*}
    \nabla^2 f(x) = \left[\begin{matrix}
        \frac{\partial^2}{\partial x_1^2} f(x)
        & \frac{\partial^2}{\partial x_1 \partial x_2} f(x)
        & \cdots & \frac{\partial^2}{\partial x_1 \partial x_n} f(x) \\
        \frac{\partial^2}{\partial x_2 \partial x_1} f(x)
        & \frac{\partial^2}{\partial x_2^2} f(x)
        & \cdots & \frac{\partial^2}{\partial x_2 \partial x_n} f(x) \\
        \vdots & \vdots & \ddots & \vdots \\
        \frac{\partial^2}{\partial x_n \partial x_1} f(x)
        & \frac{\partial^2}{\partial x_n \partial x_2} f(x)
        & \cdots
        & \frac{\partial^2}{\partial x_n^2} f(x)
      \end{matrix}
    \right].
  \end{equation*}

  \begin{enumerate}[(a)]
  \item \label{item:quadratic-gradient}
    Let $f(x) = \half x^T A x + b^T x$, where $A$ is a symmetric matrix and
    $b \in \R^n$ is a vector. What is $\nabla f(x)$?

    \answer{}{Your Answer Here}
  \item \label{item:chain-rule}
    Let $f(x) = g(h(x))$, where $g : \R \to \R$ is
    differentiable and $h : \R^n \to \R$ is differentiable.
    What is $\nabla f(x)$?

    \answer{}{Your Answer Here
    }

  \item Let $f(x) = \half x^T A x + b^T x$, where
    $A$ is symmetric and $b \in \R^n$ is a vector. What is
    $\nabla^2 f(x)$?

    \answer{}{Your Answer Here}
  \item Let $f(x) = g(a^T x)$, where $g : \R \to \R$ is continuously
    differentiable and $a \in \R^n$ is a vector.
    What are $\nabla f(x)$ and $\nabla^2 f(x)$?
    (\emph{Hint:} your expression for $\nabla^2 f(x)$ may have as few as
    $11$ symbols, including $'$ and parentheses.)

    \answer{}{Your Answer Here}
  \end{enumerate}

\newpage
% \input{gradients}
% -*- Mode: latex -*-

\item \points{0} \textbf{Positive definite matrices}

  A matrix $A \in \R^{n \times n}$ is \emph{positive semi-definite}
  (PSD), denoted $A \succeq 0$, if
  $A = A^T$ and $x^T A x \ge 0$ for all $x \in \R^n$.
  A matrix $A$ is \emph{positive definite}, denoted $A \succ 0$,
  if $A = A^T$ and $x^T A x > 0$ for all $x \neq 0$, that is,
  all non-zero vectors $x$. The simplest example of a positive
  definite matrix is the identity  $I$ (the diagonal matrix with $1$s on
  the diagonal and $0$s elsewhere), which satisfies
  $x^T I x = \ltwo{x}^2 = \sum_{i = 1}^n x_i^2$.
  \begin{enumerate}[(a)]
  \item Let $z \in \R^n$ be an $n$-vector.
    Show that $A = zz^T$ is positive semidefinite.
    
	\answer{}{Your Answer Here}

  \item Let $z \in \R^n$ be a \emph{non-zero} $n$-vector.
    Let $A = zz^T$. What is the null-space of $A$?
    What is the rank of $A$?

\answer{}{Your Answer Here}
   

  \item Let $A \in \R^{n \times n}$ be positive semidefinite and
    $B \in \R^{m \times n}$ be arbitrary, where $m, n \in \mathbb{N}$. Is
    $BAB^T$ PSD?  If so, prove it.  If not, give a counterexample with
    explicit $A, B$.

\answer{}{Your Answer Here}
  \end{enumerate}

\newpage

  \item \points{0} \textbf{Eigenvectors, eigenvalues, and the spectral
    theorem}

  The eigenvalues of an $n \times n$ matrix $A \in \R^{n \times n}$ are the
  roots of the characteristic polynomial $p_A(\lambda) = \det(\lambda I - A)$,
  which may (in general) be complex.  They are also defined as the values
  $\lambda \in \mathbb{C}$ for which there exists a vector
  $x \in \mathbb{C}^n$ such that $Ax = \lambda x$. We call such a pair
  $(x, \lambda)$ an \emph{eigenvector, eigenvalue} pair.
  In this question, we use the notation
  $\diag(\lambda_1, \ldots, \lambda_n)$ to denote the diagonal matrix with
  diagonal entries $\lambda_1, \ldots, \lambda_n$, that is,
    \begin{equation*}
      \diag(\lambda_1, \ldots, \lambda_n)
      = \left[\begin{matrix} \lambda_1 & 0 & 0 & \cdots & 0 \\
          0 & \lambda_2 & 0 & \cdots & 0 \\
          0 & 0 & \lambda_3 & \cdots & 0 \\
          \vdots & \vdots & \vdots & \ddots  & \vdots \\
          0 & 0 & 0 & \cdots & \lambda_n \end{matrix} \right].
    \end{equation*}
    
  \begin{enumerate}[(a)]

  \item
    \label{item:diagonalizable-A}
    Suppose that the matrix $A \in \R^{n \times n}$ is diagonalizable,
    that is, $A = T \Lambda T^{-1}$ for an invertible matrix $T \in \R^{n
      \times n}$, where $\Lambda = \diag(\lambda_1, \ldots, \lambda_n)$ is
    diagonal. Use the notation $t^{(i)}$ for the columns
    of $T$, so that $T = [t^{(1)} ~ \cdots ~ t^{(n)}]$, where $t^{(i)} \in \R^n$. Show
    that $A t^{(i)} = \lambda_i t^{(i)}$, so that
    the eigenvalues/eigenvector pairs of $A$ are $(t^{(i)}, \lambda_i)$.

	\answer{}{Your Answer Here}
  \end{enumerate}

  A matrix $U \in \R^{n \times n}$ is orthogonal if $U^T U = I$.
  The spectral theorem, perhaps one of the most important theorems in
  linear algebra, states that if $A \in \R^{n \times n}$ is symetric,
  that is, $A= A^T$,
  then $A$ is \emph{diagonalizable by a real orthogonal matrix}. That is,
  there are a diagonal matrix $\Lambda \in \R^{n \times n}$ and
  orthogonal matrix $U \in \R^{n \times n}$ such that
  $U^T A U = \Lambda$, or, equivalently,
  \begin{equation*}
    A = U \Lambda U^T.
  \end{equation*}
  Let $\lambda_i = \lambda_i(A)$ denote the $i$th eigenvalue of $A$.
  \begin{enumerate}[(a)]
    \setcounter{enumii}{1}
  \item Let $A$ be symmetric. Show that if $U = [u^{(1)} ~ \cdots ~ u^{(n)}]$
    is orthogonal,
    where $u^{(i)} \in \R^n$ and $A = U \Lambda U^T$, then
    $u^{(i)}$ is an eigenvector of $A$ and
    $A u^{(i)} = \lambda_i u^{(i)}$, where
    $\Lambda = \diag(\lambda_1, \ldots, \lambda_n)$.

\answer{}{Your Answer Here}
  \item Show that if $A$ is PSD, then $\lambda_i(A) \ge 0$ for each $i$.

\answer{}{Your Answer Here}
  \end{enumerate}
\newpage
% \input{matrices}
% \input{matrix-derivatives}
\item \points{0} \textbf{Probability and multivariate Gaussians}

  Suppose $X=(X_1,..X_n)$ is sampled from a multivariate Gaussian
  distribution with mean $\mu$ in $\mathbb{R}^n$ and covariance $\Sigma$ in $S^n_+$
  (i.e. $\Sigma$ is positive semidefinite).
  This is commonly also written as $X \sim \mathcal{N}(\mu, \Sigma)$.

  \begin{enumerate}[(a)]
  \item
  Describe the random variable $Y = X_1 + X_2 + \ldots + X_n$.
  What is the mean and variance?
  Is this a well known distribution, and if so, which?
  
  \answer{}{Your Answer Here}

  \item
  Now, further suppose that $\Sigma$ is invertible. Find $\mathbb{E}[X^T\Sigma^{-1}X]$.
  (Hint: use the property of trace that $x^TAx = \text{tr}(x^TAx)$).
  
  \answer{}{Your Answer Here}

  \end{enumerate}
\end{enumerate}

\end{document}
